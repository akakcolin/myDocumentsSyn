\documentclass[xcolor=x11names,UTF8]{ctexbeamer}

\graphicspath{{figures/}} % 图片路径
\usepackage{ragged2e}
\renewcommand{\raggedright}{\leftskip=0pt \rightskip=0pt plus 0cm} % 对齐
% \usepackage{calligra} % Thank you

\usepackage{natbib} % 参考文献

\mode<presentation> {

% The Beamer class comes with a number of default slide themes
% which change the colors and layouts of slides. Below this is a list
% of all the themes, uncomment each in turn to see what they look like.
\usetheme{metropolis}
%\usetheme{default}
%\usetheme{AnnArbor}
%\usetheme{Antibes}
%\usetheme{Bergen}
%\usetheme{Berkeley}
%\usetheme{Berlin}
%\usetheme{Boadilla}
%\usetheme{CambridgeUS}
%\usetheme{Copenhagen}
%\usetheme{Darmstadt}
%\usetheme{Dresden}
%\usetheme{Frankfurt}
%\usetheme{Goettingen}
%\usetheme{Hannover}
%\usetheme{Ilmenau}
%\usetheme{JuanLesPins}
%\usetheme{Luebeck}
%\usetheme{Madrid}

%\usetheme{Malmoe}
%\usetheme{Marburg}
%\usetheme{Montpellier}
%\usetheme{PaloAlto}
%\usetheme{Pittsburgh}
%\usetheme{Rochester}
%\usetheme{Singapore}
%\usetheme{Szeged}
%\usetheme{Warsaw}

% As well as themes, the Beamer class has a number of color themes
% for any slide theme. Uncomment each of these in turn to see how it
% changes the colors of your current slide theme.

%\usecolortheme{albatross}
%\usecolortheme{beaver}
%\usecolortheme{beetle}
%\usecolortheme{crane}
%\usecolortheme{dolphin}
%\usecolortheme{dove}
%\usecolortheme{fly}
%\usecolortheme{lily}
%\usecolortheme{orchid}
%\usecolortheme{rose}
%\usecolortheme{seagull}
%\usecolortheme{seahorse}
%\usecolortheme{whale}
%\usecolortheme{wolverine}

%\setbeamertemplate{footline} % To remove the footer line in all slides uncomment this line
%\setbeamertemplate{footline}[page number] % To replace the footer line in all slides with a simple slide count uncomment this line

%\setbeamertemplate{navigation symbols}{} % To remove the navigation symbols from the bottom of all slides uncomment this line
}

\usepackage{graphicx} % Allows including images
\usepackage{booktabs} % Allows the use of \toprule, \midrule and \bottomrule in tables

%----------------------------------------------------------------------------------------
%	TITLE PAGE
%----------------------------------------------------------------------------------------

\title[Short title]{【文献】Machine-learning accelerated geometry optimization in molecular simulation} % The short title appears at the bottom of every slide, the full title is only on the title page

\author{zenghui liu} % Your name
\institute[xtalpi] % Your institution as it will appear on the bottom of every slide, may be shorthand to save space
{ xtalpi\\ % Your institution for the title page

\medskip
\textit{akakcolin@163.com} % Your email address
}
\date{\today} % Date, can be changed to a custom date

\begin{document}

\maketitle

%\begin{frame}
%  \frametitle{\textbf{Outline 目录}}
%  \begin{enumerate}
%  \item 前言
%  \item 文献
%  \item 一些结果
%  \item 总结
%\end{enumerate}
%\end{frame}

\begin{frame}{前言}

  \begin{itemize}
  \item 得到最稳定(能量最低)的结构,找反应路径。
  \item 优化算法一般用共轭梯度/BFGS
  \item 速度观感:DFT:太慢了; DFTB:慢一点;MD: 快,
  \item 期望:same work in less time | more work in the same time
  \end{itemize}
\end{frame}

\begin{frame}{一些优化加速思路}

    常规的
    \begin{itemize}
        \item  给一个好的初始猜测, precondition系列
        \item  采用基于内坐标的优化(RIC)
    \end{itemize}

    流行的机器学习方法(替代模型surrogate modeling):
    \begin{itemize}
        \item Gaussian Process Regression (GPR)  高斯过程回归 ($O(n^2N^2)$)
        \item Neural network (NN) ensemble based active learning
    \end{itemize}

\end{frame}

\begin{frame}{主要内容}
  \begin{itemize}
     \item Machine learning model for the surrogate models
     \item Relaxation with active learning
     \item Results and Discussion
  \end{itemize}

\end{frame}
\begin{frame}{ML model}

    在Behler Parrinello Neural Networks(BPNNs)基础上简化的SingleNN

    \begin{figure}
    \centering
    \includegraphics[scale=0.4]{original-BPNN.png}
    %\includegraphics[scale=0.25]{SingleNN-paper.png}
    \caption{\small \textrm{BPNNs structure}}

    \end{figure}

\end{frame}

\begin{frame}{ML model}
BPNN方程指纹函数

\begin{equation}
G_{i}^{2}=\sum_{j=1}^{N_{\text {atom }}} e^{-\eta\left(R_{i j}-R_{s}\right)^{2}} \cdot f_{c}\left(R_{i j}\right) 
\end{equation}

\begin{equation}
G_{i}^{4}=2^{1-\zeta} \sum_{j, k \neq i}\left[\left(1+\gamma \cdot \cos \theta_{i j k}\right)^{\zeta} \cdot e^{-\eta\left(R_{i j}^{2}+R_{i k}^{2}+R_{j k}^{2}\right)} \cdot f_{c}\left(R_{i j}\right) \cdot f_{c}\left(R_{i k}\right) \cdot f_{c}\left(R_{j k}\right)\right]
\end{equation}

\begin{equation}
f(x)=\left\{\begin{array}{ll}
\frac{1}{2}\left(\cos \left(\frac{\pi R_{i j}}{R_{c}}+1\right)\right), & \text { for } \quad R_{i j} \leq R_{c} \\
0, & \text { otherwise. }
\end{array}\right
\end{equation}
\end{frame}

\begin{frame}{ML model}

    在Behler Parrinello Neural Networks(BPNNs)基础上简化的SingleNN

\begin{figure}
    \centering
    %\includegraphics[scale=0.25]{original-BPNN.png}
    \includegraphics[scale=0.4]{SingleNN-paper.png}
    \caption{\small \textrm{SingleNN structure}}
\end{figure}

\end{frame}

\begin{frame}{ML model}

\begin{equation}
    w G_{i}^{2}=\sum_{j=1}^{N_{a t o m}} e^{-\eta\left(R_{i j}-R_{s}\right)^{2}} \cdot f_{c}\left(R_{i j}\right) \cdot W\left(z_{j}\right) 
\end{equation}
    
\begin{equation}
    w G_{i}^{4}=2^{1-\zeta} \sum_{j, k \neq i}\left[\left(1+\gamma \cdot \cos \theta_{i j k}\right)^{\zeta} \cdot e^{-\eta\left(R_{i j}^{2}+R_{i k}^{2}+R_{j k}^{2}\right)} \cdot f_{c}\left(R_{i j}\right) \cdot f_{c}\left(R_{i k}\right) \cdot f_{c}\left(R_{j k}\right) \cdot W\left(z_{j}, z_{k}\right)\right
\end{equation}

\end{frame}

\begin{frame}{ML model}

  \begin{equation}
E_{i}=\left[\mathbf{W}^{(2)} f_{a}^{(2)}\left(\mathbf{W}^{(1)} f_{a}^{(1)}\left(\mathbf{W}^{(0)} \mathbf{g}_{i}+\mathbf{b}^{(0)}\right)+\mathbf{b}^{(1)}\right)+\mathbf{b}^{(2)}\right]_{e l_{i}}
    \end{equation}

    \begin{equation}
    E_{tot} = \sum_i^N E_i
\end{equation}

\begin{equation}
\mathbf{f}_{i}=-\frac{\partial E_{tot}}{\partial \mathbf{r}_{i}}
\end{equation}

mean square error (MSE) loss function
\begin{equation}
L=\frac{1}{N} \sum_{i}^{N}\left(E_{i}-\hat{E}_{i}\right)^{2}+\lambda \frac{1}{\sum_{i} M_{i}} \sum_{i}^{N} \sum_{j}^{M_{i}}\left(F_{i j}-\hat{F}_{i j}\right)^{2}
\end{equation}

    超参数数量减少了:BPNN 大概4446个参数,SingleNN : 846

\end{frame}
    %神经网络有2个隐藏层,每层50个神经元,激活函数tanh
    %原子环境的特征表示用基于原子中心的对称方程表示(ACSF)

\begin{frame}{ML model}
  %训练/预测;

    不确定性的衡量方式,方差近似估计:
  \begin{equation}
      T=\alpha \max _{i} \operatorname{Var}\left[E_{tot}^{i}\right]
  \end{equation}

  \begin{figure}
    \includegraphics[scale=0.35]{figure2-uncertain.png}
  \end{figure}

  %  \begin{itemize}
  %  \item With high probability, convergence of the training process is easy from a random
  %    initialization
  %    \item an overparameterized NN could lead to
%dissimilar models with high probability only using different
%random initialization
%\end{itemize}

    %如何避免/降低过拟合的问题:   early stopping

    %\begin{itemize}
    %  \item the surrogate model is only used when uncertainty is low
    %  \item if the uncertainty exceeds a threshold, the data are augmented by new DFT data
    %  \item the final minimum is always validated by DFT
    %  \end{itemize}
\end{frame}

\begin{frame}{Relaxation with active learning}
  \begin{figure}
      \centering
      \includegraphics[scale=0.3]{relax_with_online.png}
  \end{figure}

这个和大多数模型框架都很相似的

%因为不相似原子环境可以用一个通用的NN 模型来表示 同时处理多个构型,额外的加速

%Another benefit ofthe pooling is that it could be applied in a scalable way
%不同构型共享一个替代模型,不需要为每个轨迹的训练单独分配资源计算

    %Different configurations could share a common surrogate model
    %and ddthere is no need to assign separate computing
    %resources for the training of each trajectory

\end{frame}
\begin{frame}{Active learning for geometry optimization of single configuration}

  \begin{figure}
      \centering
      \includegraphics[scale=0.3]{NN-surface-compare.png}
  \end{figure}

    \begin{itemize}
        \item 替代模型需要少量的结构优化来得到 合理模型
    \end{itemize}
\end{frame}

\begin{frame}{Further accelaration by information sharing among configurations and utilizing prior data}
  \begin{figure}
      \centering
      \includegraphics[scale=0.35]{number-dft-call.png}
  \end{figure}
 acrolein/AgPd  about 193 DFT calls to converge without ML

\end{frame}

\begin{frame}{Further accelaration by information sharing among configurations and utilizing prior data}
\begin{figure}
      \centering
      \includegraphics[scale=0.35]{NN-force.png}
  \end{figure}
\end{frame}

\begin{frame}{More complex systems and NEB calculations}
\begin{figure}
      \centering
      \includegraphics[scale=0.35]{NN-cluster-compare.png}
 \end{figure}
\end{frame}

\begin{frame}{More complex systems and NEB calculations}
\begin{figure}
      \centering
      \includegraphics[scale=0.2]{NN-NEB.png}
 \end{figure}

\begin{figure}
      \centering
      \includegraphics[scale=0.3]{nn-table.png}
 \end{figure}
\end{frame}

\begin{frame}{Limiting the training data to recent configurations for training efficiency}
\begin{figure}
      \includegraphics[scale=0.25]{NN-training-time.png}
      \includegraphics[scale=0.25]{person-coeff.png}
 \end{figure}

\end{frame}

\begin{frame}{Conclusion}
  \begin{itemize}
        \item active learning with multiple configurations can
        achieve further acceleration by sharing the information across different configurations using a common NN ensemble
        \item active learning reduces the amount of DFT calls by $50-90\%$ based on systems
        \item the acceleration is even more apparent ($∼98\%$)  in CI-NEB
    \end{itemize}
\end{frame}

\begin{frame}
\Huge{\centerline{The End}}
\end{frame}

\begin{frame}
  \begin{figure}
      \centering
      \includegraphics[scale=0.3]{paper-title-neb-2019-gpr.png}
  \end{figure}

\end{frame}


\begin{frame}{GPR过程}
    高斯过程,预测
    先验均值 和核函数
    \begin{equation}
f(x) \sim G P\left(\boldsymbol{\mu}, k\left(\boldsymbol{x}, \boldsymbol{x}^{\prime}\right)\right)
\end{equation}
$\m$u 代表先验的能量和力,这里初始为0。给一个训练集
预测的均值和方差为
\begin{equation}
E[f(\boldsymbol{x} \mid D)]=\boldsymbol{k}(\boldsymbol{x})\left[\boldsymbol{K}\left(\boldsymbol{x}+\sigma_{n}^{2} \boldsymbol{I}\right)\right]^{-1} \boldsymbol{y}
\end{equation}
\begin{equation}
V[f(\boldsymbol{x} \mid D)]=k(\boldsymbol{x}, \boldsymbol{x})-\boldsymbol{k}(\boldsymbol{x})^{T}\left[\boldsymbol{K}\left(\boldsymbol{x}+\sigma_{n}^{2} \boldsymbol{I}\right)\right]^{-1} \boldsymbol{k}(\boldsymbol{x})
\end{equation}
核函数的形式为
\begin{equation}
\boldsymbol{K}(\boldsymbol{x})=\left(\begin{array}{ll}
\boldsymbol{K}_{e e}(\boldsymbol{x}, \boldsymbol{x}) & \boldsymbol{K}_{e f}(\boldsymbol{x}, \boldsymbol{x}) \\
\boldsymbol{K}_{e f}(\boldsymbol{x}, \boldsymbol{x}) & \boldsymbol{K}_{f f}(\boldsymbol{x}, \boldsymbol{x})
\end{array}\right)
\end{equation}
\end{frame}

\begin{frame}{GPR过程}
\begin{equation}
k_{e e}\left(\boldsymbol{x}, \boldsymbol{x}^{\prime}\right)=\sigma_{f}^{2} \exp \left(-\frac{1}{2} \sum_{d=1}^{D} \frac{\left(x_{d}-x_{d}^{\prime}\right)^{2}}{l_{d}^{2}}\right)
\end{equation}

\begin{equation}
\begin{array}{c}
k_{f e}\left(\boldsymbol{x}, \boldsymbol{x}^{\prime}\right)=-\frac{\sigma_{f}^{2}\left(x_{d}-x_{d}^{\prime}\right)}{l_{d}^{2}} \exp \left(-\frac{1}{2} \sum_{j=1}^{D} \frac{\left(x_{j}-x_{j}^{\prime}\right)^{2}}{l_{j}^{2}}\right) \\
k_{f f}\left(\boldsymbol{x}, \boldsymbol{x}^{\prime}\right)=\frac{-\sigma_{f}^{2}}{l_{d 1}^{2}}\left(\delta_{d_{1} d_{2}}-\frac{\left(x_{d 1}-x_{d 1}^{\prime}\right)\left(x_{d 2}-x_{d 2}^{\prime}\right)}{l_{d 1}^{2}}\right) \exp \left(-\frac{1}{2} \sum_{j=1}^{D} \frac{\left(x_{j}-x_{j}^{\prime}\right)^{2}}{l_{j}^{2}}\right)
\end{array}
\end{equation}

\end{frame}

\begin{frame}{最近几年的一些进展}

  \begin{figure}
      \centering
      \includegraphics[scale=0.2]{paper-title-neb-2017.png}
  \end{figure}

  \begin{figure}
      \centering
      \includegraphics[scale=0.2]{paper-title.png}
  \end{figure}

% 通过抽样来产生一个代

\end{frame}

\begin{frame}
% 通过抽样来产生一个代理模型,
%然后在这个模型上进行优化。
%理论上,如果这个模型足够精确的话,
%那么该模型的最优解就是求解的最优解

%    \begin{itemize}
%        \item 更多的数据
%    \end{itemize}
  结构优化:一种加速策略是 给一个比较好的初始猜测
  另一种方法是使用替代模型,该模型在计算上很便宜,但足够准确,可以在需要进行 DFT 计算之前使用便宜的模型采取许多步骤。
  提出一个在线学习的方式来加速多个结构的同时优化
  训练用于优化目标结构的替代模型,随着测试集的增加,训练时间增加,比单纯dft优化的时间还长
  这个问题如何解决:
  这是 GPR/DP方法共性问题
  用一个local trainning set  足够来解决局部优化问题

  另外一个是 主动学习的弛豫过程中的势函数

  \begin{itemize}
  \item relax from scratch
  \item relax from a small sets
  \item relax from a existing big sets

  \end{itemize}
  不同结构弛豫过程中的信息 可以彼此共享
  最后看一下这套方案 在结构优化中的性能

\end{frame}

\begin{frame}
  Machine learning model for the surrogate models

  Behler Parrinello Neural Networks(BPNN)
  基于修改版的BPNN,用一个NN网络加上不同的输出层权重来表示不同元素,
  避免了每种元素都带一个nn网络,减少了训练参数,训练时间缩短

  原子环境的特征表示用基于原子中心的对称方程表示(ACSF)

  神经网络有2个隐藏层,每层50个神经元,激活函数tanh

  数据集50个结构

    \begin{itemize}
    \item With high probability, convergence of the training process is easy from a random
      initialization
      \item an overparameterized NN could lead to
dissimilar models with high probability only using different
random initialization
\end{itemize}

    如何避免/降低过拟合的问题
    early stopping

    \begin{itemize}
      \item the surrogate model is only used when uncertainty is low
      \item if the uncertainty exceeds a threshold, the data are augmented by new DFT data
      \item the final minimum is always validated by DFT

      \end{itemize}
    \end{frame}

    \begin{frame}

  \begin{figure}
      \centering
      \includegraphics[scale=0.2]{singleNN.png}
      \caption{\small \textrm{Single NN structure}}
  \end{figure}

  \begin{itemize}
  \item SingleNN  用统一的NN来表示原子,需要的参数比BPNN少
  \item Feature: atom-centered symmetry function (ACSF)
  \item  The activation function used is tanh
  \end{itemize}
\end{frame}
\begin{frame}
   \begin{itemize}
   \item the surrogate model is only used when uncertainty is low
   \item if the uncertainty exceeds a threshold, the data are augmented by new DFT data
   \item the final minimum is always validated by DFT
   \end{itemize}
\end{frame}

\begin{frame}
  为测量模型预测的不确定性,用NN系综来做近似估计: 用10个NN,每个都有相同的结构
  直觉是,
  如果 NN 集成在测试配置上的方差与训练集中的方差相似,那么我们可以预期测试配置接近训练数据集的区域;
  因此,我们可以预期与训练误差类似的误差。
  如果离训练集中的最大方差很远,很可能发生了外推,我们要小心预测
\end{frame}

\begin{frame}
  机器学习是一个数据驱动的方法:train/predict

  训练模型 vs 结构优化?

  训练过程 可能比纯粹的结构优化更耗时/资源 

  local training dataset is ok

  考虑3中场景
  \begin{itemize}
    \item relaxation from scratch
    \item relaxation from a small dataset
    \item relaxation from a large existing dataset
  \end{itemize}

  不同优化轨迹之间的信息可以相互共享来获得额外加速效果
\end{frame}
\begin{frame}
  %数据量的问题,只做了结构优化一般跑几百个离子步就结束的问题
 %\begin{figure}
 %     \centering
 %     \includegraphics[scale=0.5]{singleNN.png}
 %     \caption{\small \textrm{Single NN structure}}
 % \end{figure}

  SingleNN  用统一的NN来表示原子,需要的参数比BPNN少
  Feature: atom-centered symmetry function (ACSF)
  The activation function used is tanh
  监测每一步的模型预测结果,如果不确定性超过阀值,就调用dft来获得准确的能量/力,来修正模型

    \begin{itemize}
    \item 原子指纹 atomic fingerprints 不同原子处在相似局域环境中的能量
    \item  如何处理不确定性
    \end{itemize}
\end{frame}

\begin{frame}

%  \item Gaussian Approximation Potentials (GAP)
%  \item  Behler Parrinello Neural Networks(BPNNs)
%  \end{itemize}

    \begin{figure}
           \includegraphics[scale=0.2]{original-BPNN.png}
    \end{figure}
    BPNN有2个主要缺点

    不同元素的指纹不兼容,体系依赖,可以做一个通用的,维度会很大

    带权重的对称方程来解决
    不同元素的原子能来自不同的NN, 泛化能力有限;

    输出层分享隐藏层的权重, 同时有元素依赖的权重,
%
    找到一个共同的指纹转换
    不同元素数据同时利用另一个数据集的非线形层
    对新数据 我只需训练输出层的线形权重, 这样可以减少训练时间和训练数据比较少

    超参数数量 BPNN ([(14 + 1) × 20 + (20 + 1) × 20 + (20 + 1)] × 6 = 4446 parameters)
    (14 + 1) × 20 + (20 + 1) × 20 + (20 + 1) × 6 = 846
\end{frame}
%----------------------------------------------------------------------------------------

\end{document}
